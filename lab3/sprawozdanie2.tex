\documentclass[11pt]{article}
\usepackage[polish]{babel}
\usepackage[utf8]{inputenc}
\usepackage[T1]{fontenc}
\usepackage{amsmath}
\usepackage{amssymb,amsfonts,amsthm}
\usepackage{multicol}
\usepackage{array}
\usepackage{geometry}
\geometry{legalpaper, margin=1in}
\usepackage{graphicx}

\title{KCK - Przetwarzanie sygnałów - rozpoznawanie płci}
\author{Bartosz Żelek 151860}

\begin{document}

\maketitle
\section{Opis programu}
W stworzonym programie zdecydowałem się na metodę analizy cepstrum.
Program rozpoczyna się od wczytania pliku z sygnałem.
Jeżeli sygnał posiada więcej niż jeden kanał, to wybierany jest pierwszy z nich.
Na danych z wybranego kanału wykonywana jest funkcja \begin{verbatim}medfilt(data, 3)\end{verbatim} w celu usunięcia szumów.
Przed przejściem z dziedziny czasu do dziedziny częstotliwości, sygnał zostaje przemnożony przez okno Flat Top w celu zminimalizowania efektu wycieku widma (zniekształcenia widma) oraz obliczane są częstotliwości, ponieważ sama operacja transformacji Fouriera zwraca zbiór liczb zespolonych, które same w sobie nie reprezentują określonych częstotliwości.
Następnie wykonywana jest transformacja Fouriera na przefiltrowanym sygnale i tym samym otrzymywane jest spektrum.
Aby otrzymać cepstrum, należy wykonać transformację Fouriera na logarytmie z modułu spektrum.
W celu wyznaczenia częstotliwości dla cepstrum ponownie wykorzystujemy funkcję \begin{verbatim}cepstrum_freqs = np.fft.fftfreq(len(log_spectrum), (freqs1-freqs[0]))\end{verbatim} - jej pierwszym argumentem jest liczba punktów w cepstrum, a w drugim różnica między kolejnymi częstotliwościami dla spektrum - rozdzielczość częstotliwości.
Następnie określany jest zakres częstotliwości, dla których szukamy maksimum. Tu zdecydowałem się na zakres od 50 Hz do 300 Hz (zakres większy niż standardowo generowany przez człowieka podczas mówienia).
Wybierane są indeksy odpowiadające tym częstotliwościom, a następnie wybierane są wartości cepstrum dla tych indeksów.
Z tych wartości wybierane jest maksimum.
Aby otrzymać częstotliwość podstawową, należy odwrócić otrzymaną wartość.
W celu wyznaczenia płci, należy porównać otrzymaną częstotliwość podstawową z wartością graniczną 170 Hz. Jeżeli otrzymana wartość jest mniejsza niż 170 Hz, to sygnał należy zaklasyfikować jako męski, w przeciwnym wypadku jako żeński.
\section{Uwagi}
Niektóre mechanizmy i wartości użyte w programie zostały wybrane drogą eksperymentalną.
Do nich należą:
\begin{itemize}
    \item Wartość 3 w funkcji \begin{verbatim}medfilt(data, 3)\end{verbatim} - wartość ta została wybrana w celu usunięcia szumów z sygnału, ale nie wpływania na jego kształt. Istnieją inne metody usuwania szumów, ale ta wydawała mi się najprostsza i najbardziej efektywna.
    \item Testowałem różne okna: Hamminga, Hanninga, Bartletta, Blackmana, Parzena, Bohmana, Blackmana-Harrisa, …, ale okno Flat Top dawało najlepsze rezultaty na danych testowych.
    \item Bez mechanizmu odszumiania okno Flat Top nie dawało najlepszych rezultatów (lepsze rezultaty dawało np. okno Blackmana).
    \item Rozszerzenie zakresu częstotliwości dla którego szukamy maksimum w cepstrum z 85-255 Hz do 50-300 Hz zmniejszyło błąd klasyfikacji.
    \item Wartość graniczna 170 Hz została wybrana drogą eksperymentalną. Iteracyjnie sprawdzałem wartości od 140 Hz do 200 Hz i znalazłem lokalne minimum błędu klasyfikacji.
\end{itemize}


\end{document}
